\chapter{绪论}

此处格式已按模板设定,作者只需选择段落区域,输入替换之。模版中所有说明性文字用于注释格式与内容的要求,撰写论文时请删除。模版中,图表、公式、参考文献等都已给出范例,撰写论文时请删除。

本模版已包含符合章节设置的“多级别列表”,只需在相应位置替换标题文字即可。如需增加章节,建议先使用格式刷,再调整编号。

\section{毕业设计(论文)类型及基本要求}

\subsection{设计类}

学生必须独立完成一定数量的设计图纸,撰写一篇15000字以上的设计说明书,工程设计类图纸折合成零号图纸不能少于三张。

\subsection{论文类}

学生必须独立完成一项研究或实验,撰写一篇15000字以上的论文。基础理论类的研究型论文的正文文字一般不少于10000字,要求内容充实,论据充分可靠,论证有力,主题明确。

\subsection{软件类}

学生必须独立完成一个软件或较大软件中的一个模块设计,撰写一篇15000字以上的设计说明书。

\section{毕业设计(论文)资料组成及存档要求}

毕业设计(论文)资料包括:任务书、开题报告、毕业设计(论文)、外文资料、中文译文、过程指导记录表、中期检查记录表、指导教师评阅书、评阅教师评阅书、答辩记录书,以及图纸、实验报告、计算程序等。

毕业设计(论文)的纸质版资料由各院、系自行安排保存,电子版资料在天津大学本科生毕业设计(论文)管理系统中保存,要保证两类资料的一致性。

毕业设计(论文)及图纸、实验报告、计算程序等装订成册(图纸数量过多可单独装订成册),附件资料按顺序装订成册(资料顺序为:附件封面、目录、任务书、开题报告、外文资料、中文译文、过程指导记录表、中期检查记录表、指导教师评阅书、评阅教师评阅书、答辩记录书),两册一并存档,两册封面均采用天津大学本科生毕业设计(论文)统一封面。

\chapter{理论研究}

此处格式已按模板设定,作者只需选择段落区域,输入替换之。模版中所有说明性文字用于注释格式与内容的要求,撰写论文时请删除。模版中,图表、公式、参考文献等都已给出范例,撰写论文时请删除。

本模版已包含符合章节设置的“多级别列表”,只需在相应位置替换标题文字即可。如需增加章节,建议先使用格式刷,再调整编号。

\section{毕业设计(论文)结构及要求}

\subsection{论文结构}

毕业设计(论文)原则上应采用中文完成(教学语言为英语的除外),确需用其他语言撰写的需经学院审批同意并报教务处备案。毕业设计(论文)一般由以下部分组成,依次为:①封面;②扉页;③独创性声明;④中英文摘要及关键词;⑤目录;⑥正文;⑦参考文献;⑧附录;⑨致谢。

\subsection{语言表述}

要做到数据可靠、推理严谨、立论正确。论述必须简明扼要、重点突出,对同行专业人员已熟知的常识性内容,尽量减少叙述。

论文中如出现一些非通用性的新名词、术语或概念,需做出解释。

\subsection{标题和层次}

标题要重点突出,简明扼要,层次要清楚。

\subsection{打印规格}

论文一律采用A4纸张(大小为210mm$\times$297mm)打印,可根据实际选择单面或双面印刷,页边距如下设置。上:27.5mm;下:25.4mm;左:35.7mm;右:27.7mm。页眉距边界15.0mm;页脚距边界17.5mm。字符间距为默认值(缩放100\%,间距:标准)。


\section{毕业设计(论文)撰写规范}

\subsection{封面}

采用天津大学本科生毕业设计(论文)统一封面,封面内容包括论文题目、学院、专业、年级、姓名、学号、指导教师等信息。

论文题目是论文总体内容的体现,要求醒目、简明、准确,主题突出,一般不宜超过25字。

\subsection{目录}

目录的各章节应简明扼要,应列至三级标题,包含正文及其后的各部分,并附有相应页码。目录的文字应与相应标题文字完全一致。

“目录”两字之间空一个全角空格或两个半角空格,采用不编号章标题样式。目录条目采用正文样式。

各级标题采用逐级缩进形式,每级缩进2字符,页码前导符采用“…”。

\subsection{正文}

正文是毕业设计(论文)的主体,应占据主要篇幅,文字一般不少于15000字,要求主题明确,内容充实;论点正确,论据可靠,论证充分。内容一般包括:设计(论文)的工作目的(背景),国内外研究现状、理论分析、计算方法、实验装置和测试方法、实验结果分析与讨论、研究成果、结论及意义等。

正文中文字体为宋体,英文字体为Times New Roman,正文采用小四号字,段落行间距为固定值20磅,段落前后间距为0,首行缩进2字符。西文字体以Times New Roman为准,若Times New Roman中没有相应字符,则应使用较为清晰和通用的字体。数学公式和专门文字(如计算机程序代码)的字体可以根据需要选择。

章节与标号:一般分为章标题(一级标题)、不编号章标题(同属于一级标题)、二级标题和三级标题。各章节编号建议采用Word的“多级别列表”方式自动形成编号,标题编号与标题内容之间空一个全角空格或两个半角空格。各级章节标题格式要求细节参见《天津大学本科生毕业设计(论文)撰写规范》。

图、表等与其前后的正文之间要有一行的间距;文中的图、表、公式一律采用阿拉伯数字分章编号,如:图2-5,表3-2,公式(5-1)(“公式”两个字不要写上)等。若图或表中有附注,采用英文小写字母顺序编号。子图采用英文字母编号。引用图或表应在图题或表题右上角标出文献来源。图或表的附注应位于图或表的下方。

\subsection{公式}

公式要标准、通用,公式的变量和参数,除了公知公认的以外,一般要给出解释。文中的公式一律采用阿拉伯数字分章编号,如:公式\eqref{eq:label}(“公式”两个字不要写上)。公式编号只能用于行间公式。公式编号右对齐,且应加英文小括号,不加引导符。其余样式与正文相同。变量和参数,不能采用正文格式,必须正确使用数学格式或行内公式格式,包含行内公式的段落可以采用单倍行距。

依照以上标准的行间公式范例如下。
\begin{align}
\label{eq:label}
	\left\{\begin{array}{l}
        F_{\mathrm{s} i}=k_{\mathrm{s} i}\left(z_{i}^{\prime}-z_{i}\right)+c_{s i}\left(\dot{z}_{i}^{\prime}-\dot{z}_{i}\right) \\
        F_{\mathrm{t} i}=k_{\mathrm{t} i}\left(z_{i}-y_{i}\left(x_{i}, t\right)\right)+c_{\mathrm{t} i}\left(\dot{z}_{i}-\dot{y}_{i}\left(x_{i}, t\right)\right)
        \end{array}\right.
\end{align}

其中:zi为车辆第i个轮胎由静平衡位置起算的竖向位移;yi为桥梁在第i个轮胎作用下的瞬时变位。

考虑整个供应链的利润函数$\beta_{SC}$。因为$\frac{\partial\beta_{SC}}{\partial p_1}=q-\int_0^q F(x)\ud x>0$,所以$\beta_{SC}$对$p_1$单调递增,所以:
\begin{equation}
\label{dscNoStgProof0}
\beta_{SC}(q_s,p_{1s},p_{2s})<\beta_{SC}(q_s,p_{1n},p_{2n})
\end{equation}

因为对于$\forall q\in[q_s, q_n)$,有:
\[ \left.\frac{\partial \beta_{SC}}{\partial q}\right|_{(q,p_{1n},p_{2n})}=p_{1n}-c+c_L+(p_{2n}-p_{1n}-c_L)F(q) \]

销售商决策如式~\eqref{rcond}~所示:
\begin{equation}
\label{rcond}
\left\{\begin{array}{l}
p_{1s}=v_h-(v_h-p_2)\mathbb{E}(\varphi) \\
p_{2s}=v_l \\
q_s \in \underset{q \geq 0}{\mathrm{argmax}} \beta_R (q, p_1, p_2) \\
\end{array}\right.
\end{equation}

\subsection{图}

图要精选、简明,切忌与表及文字表述重复。图中的术语、符号、单位等应同文字表述一致。图与其前后的正文之间要有一行的间距;文中的图一律采用阿拉伯数字分章编号,如:图2-5。若图中有附注,采用英文小写字母顺序编号。子图采用英文字母编号。图序及图题居中置于图的下方,图、表内容为五号字,中文字体为宋体,英文字体为Times New Roman。图序与图题之间空一个全角空格或两个半角空格。引用图应在图题右上角标出文献来源。图的附注应位于图或表的下方。

依照以上标准的插图范例如下。


\begin{figure}[!htbp]
  \centering
  \includegraphics[width=0.6\textwidth]{figures/tjuname.png}
  \caption{晶体坐标系和样品坐标系示意图}
  \label{fig:sample1}
\end{figure}

\begin{figure}[!htbp]
  \centering
  \subfloat[云图;]{
    \includegraphics[height=0.38\textwidth]{figures/tjulogo.png}
  }
  \subfloat[沿深度方向分布曲线]{
    \includegraphics[height=0.38\textwidth]{figures/tjulogo.png}
  }
  \caption{应变硅横截面样品的拉曼类硅峰峰位}
  \label{fig:sample2}
\end{figure}


\subsection{表}


表中参数应标明量和单位的符号。表的编排建议采用国际通行的三线表。表与其前后的正文之间要有一行的间距;文中的表一律采用阿拉伯数字分章编号,如:表3-2。若表中有附注,采用英文小写字母顺序编号。表序及表题居中置于表的上方。如某个表需要转页接排,在随后的各页上应重复表序,后跟表题(可省略)和“(续)”,置于表格上方。续表均应重复表头。表序与表题之间空一个全角空格或两个半角空格。引用表应在表题右上角标出文献来源。表的附注应位于图或表的下方。

依照以上标准的三线表范例如\ref{tab:tab1}。

\begin{table}[!htbp]
  \centering
  \caption{典型微尺度力学实验方法的基本信息}
  \label{tab:tab1}
  \vspace{0.5em}
  \begin{tabular}{ccc}
    \toprule
    \textbf{实验方法} & \textbf{主要测量对象} & \textbf{空间分辨率}    \\
    \midrule
    原子力显微镜        & 表面力             & 0.1 nm {[}1{]}    \\
    透射电镜          & 晶格结构、位错         & 0.1 nm {[}2{]}    \\
    X射线衍射         & 应变              & 1 $\mu$m{[}3-5{]}     \\
    同步辐射          & 内部三维结构与变形       & 约100 nm{[}6, 7{]} \\
    显微拉曼          & 应变              & 250 nm {[}8{]}    \\ 
    \bottomrule
  \end{tabular}
\end{table}

表格支持合并单元格等操作如\ref{tab:bigtable}

\begin{table}[htbp]
    \caption{一个很大很大的表格}
    \label{tab:bigtable}
    \centering
    \vskip 0.1mm \setlength{\tabcolsep}{1.5mm}
    \begin{tabular}{@{\extracolsep\fill}ccccc}
    \toprule[1.5pt]
    Col 1 & \multicolumn{2}{c}{Col 2} & Data 1 & Data 2 \\ 
    \midrule[1pt]
    \multirow{4}{*}{\makecell[c]{行合并}} & \multicolumn{2}{c}{列合并} & 这是很长很长很长的一段话 & 好学生\\
    \cmidrule{2-5}
    ~ & \multicolumn{2}{c}{列合并} & 6 & 学生 \\
    \cmidrule{2-5}
    ~ & \multicolumn{2}{c}{列合并} & 3 & 学生\\
    \cmidrule{2-5}
    ~ & \multicolumn{2}{c}{列合并} & 3 & 好学生\\
    \midrule[1pt]
    \multirow{8}{*}{\makecell[c]{行合并}} & \multirow{2}{*}{没有} & 合并 & 3 & 学生\\
    ~ & ~ & 合并 & 3 & 学生 \\
    \cmidrule{2-5}
    ~ & \multirow{2}{*}{没有} & 合并 & 3 & 学生\\
    ~ & ~ & 合并 & 3 & 学生 \\
    \cmidrule{2-5}
    ~ & \multirow{2}{*}{没有} & 合并 & 3 & 学生\\
    ~ & ~ & 合并 & 3 & 学生 \\
    \cmidrule{2-5}
    ~ & \multirow{2}{*}{没有} & 合并 & 3 & 学生\\
    ~ & ~ & 合并 & 3 & 学生 \\
    \toprule[1.5pt]
    \end{tabular}
\end{table}


\section{脚注}

在正文当中\footnote{脚注字号为五号,其余样式与正文相同。},如果有个别名词或情况需要解释时,可加注释说明。注释说明应采用文中编号加脚注的模式。脚注应采用阿拉伯数字上标,字体为Times New Roman,分页连续标号。脚注字号为五号,其余样式与正文相同。


\section{页眉和页脚}

页眉从正文开始到毕业设计(论文)结尾,一律设为“天津大学XX届本科生毕业设计(论文)”,采用五号字居中书写,其余样式与正文相同。

页脚从中文摘要开始,从 I 开始顺序编页码,为大写罗马数字格式。正文第一页开始,用阿拉伯数字重新从1开始顺序编页码,至学位论文结尾。页脚的字号为小五号,居中书写。

\section{参考文献}

只列出作者直接阅读过且在正文中被引用过的文献资料,本专业教科书一般不能作为参考文献,除个别专业外,一般应有外文参考文献。参考文献一律放在论文参考文献部分中,不应放在章节末尾或页面脚注中。参考文献一般不应少于15篇。“参考文献”四字采用不编号章标题样式。内容样式与正文相同。

根据GB/T 7714—2015的要求书写参考文献,可采用顺序编码制,也可采用著者-出版年制,但全文必须统一。参考文献引用使用上标引用。作者的著录方法如下:

(1)3人以下全部著录,3人以上可只著录前3人,后加“,等”,外文用“, et al ”“ et al ”不必用斜体;

(2)责任者之间用“,”分隔;

(3)责任者姓名一律采用姓前名后的著录形式。

参考文献中的标点符号:中文文献采用中文、全角、英文标点输入法输入,标点后接排后续内容;英文文献采用英文、半角、英文标点输入法输入,标点后空一格编排后续内容。

这是一篇中文参考文献\cite{RJXB202110009,zhou2016MLbook},这是一篇英文参考文献\cite{kiymik2005FT}

\section{代码环境}

很多和计算机专业背景相关的同学都会使用到代码环境,这里是一行行内代码示例\verb|printf("Hello, world!\n");|。使用verbatim环境或者是\verb|\verb|指令固然是一种选择,但是比不上专门的lstlisting环境这么专业。

\begin{lstlisting}
int main(int argc, char ** argv) {
    printf("Hello world!\n");
    return 0;
}
\end{lstlisting}


\chapter{实验与分析}

\section{任务书}
任务书内容应包括原始依据、参考文献、设计内容和要求,其中原始依据要填写明确,原始依据不得少于200字,包括设计(论文)的工作基础、研究条件、应用环境、工作目的;设计(研究)内容和要求不得少于200字,包括设计(研究)内容、主要指标与技术参数,并根据课题性质对学生提出具体要求。

\section{开题报告}
开题报告要求不少于2000字,内容包括:课题的来源及意义,国内外发展状况,本课题的研究目标、研究内容、研究方法、研究手段和进度安排,实验方案的可行性分析和已具备的实验条件以及主要参考文献等。

\section{评阅书}

\subsection{指导教师评阅书}

指导教师评阅书中的“评阅意见” 不能少于200字,主要包括对开题报告、设计或研究内容、外文资料和译文、工作量、工作态度、设计或论文质量、创新性、应用性、论文写作、文本规范、存在的不足和综合评价等方面的评阅意见,应体现对所评阅论文的具体意见,要有针对性。

\subsection{评阅教师评阅书}

评阅教师评阅书中的“评阅意见”不能少于200字,主要包括对选题、设计或研究内容、外文资料和译文、工作量、设计或论文质量、创新性、应用性、论文写作、文本规范、存在的不足和综合评价等方面的评阅意见,应体现对所评阅论文的具体意见,要有针对性。

\section{答辩记录书}

答辩记录书中的“综合评价”不能少于100字,主要包括设计或研究内容、工作量、设计或论文质量和答辩情况;“答辩记录”主要包含答辩委员提出的问题和学生回答情况等,答辩记录由答辩小组秘书填写。

\section{外文资料和中文译文}

外文资料要与所做课题紧密联系,严禁抄袭有中文译本的外文资料,外文资料的选取要注明出处。可用 A4纸复印,如果打印,标题应采用不编号章标题样式,内容应采用正文样式。中文译文的字数一般为$5000\sim 6000$汉字。